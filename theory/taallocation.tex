%% LyX 2.3.3 created this file.  For more info, see http://www.lyx.org/.
%% Do not edit unless you really know what you are doing.
\documentclass[12pt,letterpaper,reqno,english]{amsart}
\usepackage[T1]{fontenc}
\synctex=-1
\usepackage{babel}
\usepackage{amstext}
\usepackage{amsthm}
\usepackage[unicode=true,pdfusetitle,
 bookmarks=true,bookmarksnumbered=false,bookmarksopen=false,
 breaklinks=false,pdfborder={0 0 1},backref=false,colorlinks=false]
 {hyperref}

\makeatletter

%%%%%%%%%%%%%%%%%%%%%%%%%%%%%% LyX specific LaTeX commands.
\pdfpageheight\paperheight
\pdfpagewidth\paperwidth


%%%%%%%%%%%%%%%%%%%%%%%%%%%%%% Textclass specific LaTeX commands.
\theoremstyle{definition}
 \newtheorem{example}{\protect\examplename}

%%%%%%%%%%%%%%%%%%%%%%%%%%%%%% User specified LaTeX commands.
\usepackage{tikz}
\usepackage{textgreek, enumitem}
\usepackage[T1]{fontenc}
\usetikzlibrary{decorations.pathreplacing, calc,fit,shapes, positioning,arrows, patterns}


%
\@ifundefined{definecolor}
 {\usepackage{color}}{}
\usepackage{amsfonts}\setcounter{MaxMatrixCols}{30}
\usepackage{ifpdf} % part of the hyperref bundle
\ifpdf % if pdflatex is used
 % set fonts for nicer pdf view
 \IfFileExists{lmodern.sty}
  {\usepackage{lmodern}}{}
\fi % end if pdflatex is used


\providecommand{\U}[1]{\protect \rule{.1in}{.1in}}


\providecommand{\possessivecite}[1]{\citeauthor{#1}'s\nolinebreak[2]
(\citeyear{#1})}


\renewcommand{\baselinestretch}{1.35}
\setlength{\topmargin}{0.0in}
\setlength{\textheight}{8.0in}
\setlength{\evensidemargin}{0.20in}
\setlength{\oddsidemargin}{0.20in}
\setlength{\textwidth}{6.1in}


\usepackage{bbold}
\DeclareMathOperator{\EX}{\mathbb{E}}
\DeclareMathOperator{\E}{\mathbb{E}}
\DeclareMathOperator{\Prob}{\mathbb{P}}
\DeclareMathOperator{\R}{\mathbb{R}}
\DeclareMathOperator{\N}{\mathbb{N}}
\DeclareMathOperator{\1}{\mathbb{1}}
\DeclareMathOperator{\0}{\mathbb{0}}

\DeclareMathOperator{\CU}{\mathcal{U}}
\DeclareMathOperator{\CG}{\mathcal{G}}
\DeclareMathOperator{\CQ}{\mathcal{Q}}

\DeclareMathOperator{\val}{val}
\DeclareMathOperator{\marg}{marg}
\DeclareMathOperator{\supp}{supp}

\newcommand{\euler}{\mathrm{e}}

\usepackage{setspace}
%\onehalfspacing

\makeatother

\providecommand{\examplename}{Example}

\begin{document}
\title{TA Assignment Problem}
\author{Marcin Pęski}
\begin{abstract}
The goal is to formulate a TA assignment problem and discuss some
problems with tryig to solve it using stability-like notions. 
\end{abstract}

\date{\today}
\maketitle

\section{Model}

Let $S$ be the set of students (i.e., candidate TAs). Let $C$ be
the set of courses. A (TA) assignment is a mapping $\phi:C\rightarrow\R_{+}^{S}$
We refer to $\phi\left(c\right)$ as the assignment to course $c$.
If $\phi$ is an assignment, then for each $s$, define $\phi\left(s\right)\in\R_{+}^{C}$
by $\phi\left(s,c\right)=\phi\left(c,s\right)$. We refer to $\phi\left(s\right)$
as the assignment of student $s$. Let $\Phi$ be the space of the
assignments.

Let $A_{S}=\R_{+}^{S}$ be the space of possible assignments to courses.
Analogously, let $A_{C}=\R_{+}^{C}$ be the space of possible assignments
for students. 

Each course $c$ has a subset of assignments $F_{c}\subseteq A_{S}$
that are feasible for it. Examples of feasibility constraints include:
\begin{itemize}
\item budget: Each course has a budget of $h_{c}$ TA hours that need to
be allocated and assignment $\phi\left(c\right)\in A_{S}$ is feasible
for $c$ if and only if 
\[
\sum_{s}\phi\left(c,s\right)=h_{c},
\]
\item minimum assignment hours: there is a constraint $\tau_{c}$ such that
an assignment is feasible only if for each $s$, either $\phi\left(c,s\right)=0$
or $\phi\left(c,s\right)\geq\tau_{c}$. 
\end{itemize}
Similarly, each student $s$ has a subset $A_{s}\subseteq A_{C}$
of assignments that are feasible for them. Examples of student feasibility
constraints:
\begin{itemize}
\item maximum assignment: each student $s$ cannot work for more than $\kappa_{s}$
hours: 
\[
\sum_{c}\phi\left(c,s\right)\leq\kappa_{s},
\]
\item minimum assignment: student $s$ must work for at least $l_{s}$ hours,
\item balance across semesters: There is a set $M$ of ``semesters'' and
each course $c$ is associated with a subset $M_{c}\subseteq M$ of
``semesters'' in which this course is taught. The student $s$ assigned
hours in semester $m$ are equal to 
\[
h_{s}\left(m;\phi\right)=\sum_{c:m\in M_{c}}\frac{1}{\left|M_{c}\right|}\phi\left(c,s\right).
\]
It is required that the student assignment across semesters is balanced:
for some constant $\rho>1$, for each $m,m^{\prime}$, 
\[
\frac{h_{s}\left(m,\phi\right)}{h_{s}\left(m^{\prime},\phi\right)}\leq\rho.
\]
\end{itemize}
An assignment $\phi$ is \emph{feasible} if $\phi\left(i\right)\in A_{i}$
for each $i\in S\cup C$. 

Each student $s$ has preferences over possible assignments represented
by utility function $u_{s}:A_{C}\rightarrow\R$. Similarly, each course
$c$ has preferences over assignments represented by $u_{c}:A_{S}\rightarrow\R$.

An assignment is individually rational (for students) if none of the
students is better off by dropping some courses from its assignment
(the logic is that students cannot be made to work for a course, and
a typical TA contract is written separately for each course. If TAs
sign up for a bundle, i.e., for each of their assignment, the individual
rationality would require that each student prefers their assignment
to \textbf{0} assignment). Formally, for each $s$, each $D\subseteq C$,
$u\left(\phi\left(s\right)\right)\geq u\left(\phi_{D}\left(s\right)\right)$,
where $\phi_{D}$ is an assignment obtained from $\phi$ by zeroing
all courses not in $D$ $\phi_{D}\left(c,s\right)=\begin{cases}
\phi\left(c,s\right) & c\in D\\
0 & \text{otherwise}
\end{cases}$. 

Let $\Phi^{*}$ be the set of feasible and individually rational assignments. 

A social planner has preferences over $\Phi^{*}$. 

\section{Core and stability}

Suppose that the social planner would like its assignment to be stable,
in some sense. It is not clear what exactly stable assignment means
- as there are many options available. Perhaps the most straightforward
would be t consider individually rational assignments that are not
blocked by 2-agent (student and course) coalitions. But, with many-to-many
assignments, 2-agent coalitions may not be sufficient and one may
want to think about some sort of core-type concept. 

Instead of focusing on a stability notion, I want to illustrate the
problem with few examples why the any stability-like notion can be
problematic. 

\subsection{No externalities\label{subsec:No-externalities}}

Suppose that student preferences are additive: for each $s$, there
exist constants $\nu_{s}\left(c\right)$ such that for each $a\in A_{C}$,
\[
u_{s}\left(a\right)=\sum_{c}a_{c}\nu_{s}\left(c\right).
\]
Additionally, suppose that all assignments of student $s$ are subject
to maximum constraint $\kappa_{s}$. 

Similarly, suppose that course preferences are additive: for each
$c$, there exist constants $\nu_{c}\left(s\right)$ such that for
each $a\in A_{S}$,
\[
u_{c}\left(a\right)=\sum_{s}a_{s}\nu_{c}\left(s\right).
\]
Additionally, suppose that all assignments of course $c$ are subject
to maximum constraint $h_{s}$. 

In this environment, one can show that the core is non-empty. Moreover,
core allocations can be found by a version of Gale-Shapley algorithm.
(Should I add this?)

\subsection{Courses that span across semesters}

It is well-known that any type of externality (or, more precisely,
complementarities) in many-to-many matching problems leads to problems
with the existence of a stable outcome (or core).

One source of externality that reduces the use of stability as a satisfactory
solution concept is given by courses that span across multiple semesters. 
\begin{example}
\label{exa:There-are-three}There are three students $x,y,z$, and
four courses $aF,aS,bY,cY$. Courses $bY$ and $cY$ are offered across
two semesters, the other two courses are offered only in one semester.
Each course needs exactly 1 TA, and each student wants at most 1 course
per semester. All allocations must be either 1 or 0. (In this example,
partial allocations would not help anyway.) Student preferences over
bundles of courses are 
\begin{align*}
x: & \left\{ aF,aS\right\} >bY>aF>\emptyset,\\
y: & aF>cY>\emptyset,\\
z: & cY>aS>\emptyset,
\end{align*}
Essentially, student $x$ really likes course $aF$, but also needs
money and would rather TA for $aF$ and $aS$ than for the whole-year
course $Y$, but is willing to TA for $aF$ only if that's the only
option. Course preferences are 
\begin{align*}
aF: & x>y>\emptyset\\
aS: & z>x>\emptyset,\\
bY: & x>\emptyset,\\
cY: & y>z>\emptyset.
\end{align*}
So, $x$ and $aF$ are top matches for e\textbackslash each other,
but otherwise, there are no obvious matches. 

We show that there is no core. If $y$ is assigned to $aF$, then
$z$ must be assigned to $cY$ (as that is the top assignment for
$z$ and the second best assignment for $cY$ after $y$ is taken).
But then, given that the best remaining for $aS$ is $x$, $x$ and
pair $\left\{ aF,aS\right\} $ are a blocking coalition (as the student
and all courses prefer to rematch with each other). 

If $y$ is assigned to $cY$, then $z$ must be assigned with $aS$
(as $z$ is the top choice for $aS$ and $aS$ is the top choice for
$z$ aparat from $cY$ which prefers $y$. But then, since $aS$ prefers
$z$, $bY$ is the best option for $x$. As a consequence $aF$ is
available for $y$ who prefers it to $cY$.

If $y$ is not assigned to anybody, then $z$ gets $cY$, but then
$y$ and and $cY$ is a blocking pair. 
\end{example}

\subsection{Preferences over the number of assignments}

Another source of problematic externality is when either students
(or, resp., courses) have preferences over the total number of assigned
courses (resp. students). In fact, a reinterpretation of Example \ref{exa:There-are-three}
illustrates a difficulty:
\begin{example}
There are three students $x,y,z$, and four courses $a,a^{\prime},b,c$.
Each student has up to 2h of work available. Courses $b$ and $c$
require 2h of work and they only want to hire the same TA. Courses
$a,a^{\prime}$ require 1h of work each. Otherwise, the preferences
are similar to Example \ref{exa:There-are-three}: for students 
\begin{align*}
x: & \left\{ a,a^{\prime}\right\} >b>a>\emptyset,\\
y: & a>c>\emptyset,\\
z: & c>a^{\prime}>\emptyset,
\end{align*}
and for courses:
\begin{align*}
a: & x>y>\emptyset\\
a^{\prime}: & z>x>\emptyset,\\
b: & x>\emptyset,\\
c: & y>z>\emptyset.
\end{align*}
There is no stable matching.
\end{example}

\subsection{Minimum assignment}

There are $n\geq1$ students and $n+1$ courses. Each student has
$n+1$ hours and each course needs $n$ hours. The student $s$ utility
from assignment is linear in assignments (like in Section \ref{subsec:No-externalities}),
with coefficients equal to 
\[
v_{s}\left(c\right)=\begin{cases}
1+\varepsilon & c=s\\
1 & c\neq s.
\end{cases}.
\]
Course $c$ utility from the assignment $a\in A_{S}$ depends on the
number of non-zero assignments and it equal to 
\[
u_{c}\left(a\right)=\varepsilon1_{s=c}+\begin{cases}
1 & \left|\left\{ s:a_{s}>0\right\} \right|=1\\
1-\varepsilon & \left|\left\{ s:a_{s}>0\right\} \right|=2\\
1-10\left|\left\{ s:a_{s}>0\right\} \right| & \left|\left\{ s:a_{s}>0\right\} \right|>2.
\end{cases}
\]
The first term is a small boost of the utility when the student has
the same index as he course. 

Then, the unique core allocation (or stable matching under any reasonable
deifnition) is when each student is matched for $n$ hours with the
course with the same index and for remaining $1$ hours is assigned
to course $n+1$. 

Of course, the above assignment is Pareto-optimal, but wildly inefficient. 
\end{document}
